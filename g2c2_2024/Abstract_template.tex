\documentclass[a4paper,14pt]{article}
\usepackage[cp1251]{inputenc}
%\usepackage{floatflt}
\usepackage{float}
\usepackage{cite}
\usepackage[english,russian]{babel}
\usepackage{amssymb,amsmath,amsfonts}
\usepackage{color}
\usepackage{enumerate}
\usepackage[dvips]{graphicx}
\graphicspath{{C:\Users\User\Desktop\Folder}}
\usepackage{setspace}
\usepackage{xcolor}
\usepackage{fancyhdr}

\textheight=220mm
\textwidth=160mm
\oddsidemargin=0.1in
\evensidemargin=0.1in

\begin{document}
\renewcommand{\refname}{\small{References}}
\thispagestyle{fancy}
\fancyhead{}
\fancyhead[L]{\footnotesize{Graphs and Groups, Complexity and Convexity}}
\fancyhead[R]{\footnotesize{Abstracts}}
\fancyfoot{}
\fancyfoot[L]{\footnotesize{Shijiazhuang, China}}
\fancyfoot[R]{\footnotesize{August 11-25, 2024}}
\renewcommand{\footrulewidth}{0.1 mm}

\begin{center}
\textbf{The title of your abstract}\\
\vspace{\baselineskip}
Name Surname\\
\emph{Your affiliations}\\e-mail@sjtu.edu.cn
\vspace{\baselineskip}
\end{center}
\vspace{\baselineskip}

The text of your abstract.

An example of citation \cite{Pi1}. \\
\\
\begin{thebibliography}{99}
\bibitem{Pi1}
\small{Maru\v{s}i\v{c}, Dragan and Scapellato, Raffaele, Permutation groups, vertex-transitive digraphs and semiregular automorphisms. {\it European J. Combin.} \textbf{19} (1998) 707-712.}

\end{thebibliography}


\end{document} 
